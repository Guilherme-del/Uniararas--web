Que tipo de erros podem ocorrer quando não há uma
resposta do servidor “HTTP Response”? Pesquise o
handshake do serviço HTTP.

R:Quando não há uma resposta do servidor "HTTP Response", isso pode indicar que houve uma falha na comunicação entre o cliente e o servidor. Essa falta de resposta pode ser causada por vários fatores, incluindo:
-Problemas de conectividade: se o cliente não estiver conectado à internet ou se houver um problema de conexão, o servidor pode não ser capaz de enviar uma resposta.
-Problemas do servidor: se o servidor estiver sobrecarregado ou com problemas técnicos, ele pode não ser capaz de enviar uma resposta.
-Erros do cliente: se o cliente não enviar uma solicitação corretamente formatada ou se houver um problema com o navegador ou aplicativo usado, isso pode impedir que o servidor envie uma resposta.

O "handshake" do serviço HTTP refere-se ao processo de estabelecimento de uma conexão 
entre o cliente e o servidor. 
Durante esse processo, o cliente envia uma solicitação de conexão ao servidor, 
e o servidor responde com uma mensagem de confirmação de que a conexão foi estabelecida.

O handshake do HTTP envolve três etapas:
O cliente envia uma solicitação de conexão ao servidor.
O servidor responde com uma mensagem de confirmação de que a conexão foi estabelecida.
O cliente responde à mensagem de confirmação do servidor, indicando que a conexão está pronta para uso.

Se uma dessas etapas não for concluída corretamente, 
pode ocorrer uma falha na comunicação entre o cliente e o servidor, 
resultando em uma falta de resposta do servidor. 
É importante que o handshake seja executado corretamente para garantir uma comunicação 
adequada entre o cliente e o servidor.