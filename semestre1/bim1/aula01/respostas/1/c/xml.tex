Quais são as vantagens, desvantagens e necessidades de uso para XML:
R:
Vantagens:
  -Flexibilidade: o XML é altamente flexível e pode ser usado para representar uma ampla variedade de informações estruturadas.
  -Interoperabilidade: o XML pode ser lido e processado por diferentes sistemas de software, independentemente da plataforma ou linguagem de programação usada.
  -Legibilidade humana: o XML pode ser facilmente lido e compreendido por humanos, o que o torna útil para documentação e comunicação entre equipes de desenvolvimento.
Desvantagens:
  -Complexidade: a sintaxe do XML pode ser complexa, especialmente quando comparada a outras linguagens de marcação, como HTML.
  -Tamanho do arquivo: o XML pode gerar arquivos grandes devido à sua estrutura de marcação. Isso pode tornar o processamento e a transmissão desses arquivos mais lentos.
  -Não é adequado para todas as necessidades: o XML pode ser excessivo para documentos simples ou de dados que não requerem uma estruturação complexa.
Necessidades de uso:
  -Armazenamento e transmissão de dados estruturados: o XML é útil para armazenar e transmitir dados estruturados, especialmente em ambientes onde diferentes sistemas de software precisam compartilhar informações.
  -Integração de aplicativos: o XML é comumente usado para integrar aplicativos de software, permitindo que eles se comuniquem e compartilhem informações de forma consistente.
  -Troca de informações: o XML pode ser usado para trocar informações entre diferentes equipes, departamentos ou organizações, especialmente em cenários em que a estrutura e a precisão dos dados são críticas.